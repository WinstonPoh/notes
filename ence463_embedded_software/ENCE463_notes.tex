\documentclass[12pt]{article}

\title{\textbf{ENCE463: Embedded Software Engineering}}
\author{Winston Poh\\{\LaTeX} }
\date{\today}
\begin{document}

\maketitle

\section{Lectures with Steve Weddell}
\subsection{Projects}
\subsubsection{FreeRTOS with Stellaris (30\%)}

\subsection{Summary}
\begin{itemize}
	\item \textbf{Labelled Transition System Analyser (LTSA)}: A verification tool for concurrent systems. Mechanically check concurrent system specifications meet properties required of its behaviour.
	\item \textbf{Finite State Processes (FSP) Language}: In this context a language that XXX
\end{itemize}


Support for two high-level building systems, \emph{rubber}\footnote{https://launchpad.net/rubber/} \& \emph{latexmk}\footnote{http://www.phys.psu.edu/{\textasciitilde}collins/software/latexmk-jcc/} has been added to this release as well. Your preferred typesetter can be configured through the Compilation tab in the Preferences menu. Typesetters that are not installed on your system will not be selectable. 

Added for your viewing convenience is a continuous preview mode for the PDF. This mode is enabled by default, but can also be disabled through the \emph{(View $\rightarrow$ Page layout in preview)} menu. Complementary to this feature is SyncTeX integration, which allows you to synchronize the position in your editor with the PDF preview. 

\section{Feedback}
We hope you will enjoy using this release as much as we enjoyed creating it. If you have comments, suggestions or wish to report an issue you are experiencing - contact us at: \emph{http://gummi.midnightcoding.org}.

\section{One more thing}
If you are wondering where your old default text is; it has been stored as a template. The template menu can be used to access and restore it. 

\end{document}
